%%=============================================================================
%% Conclusie
%%=============================================================================

\chapter{\IfLanguageName{dutch}{Conclusie}{Conclusion}}
\label{ch:conclusie}

% TODO: Trek een duidelijke conclusie, in de vorm van een antwoord op de
% onderzoeksvra(a)g(en). Wat was jouw bijdrage aan het onderzoeksdomein en
% hoe biedt dit meerwaarde aan het vakgebied/doelgroep? 
% Reflecteer kritisch over het resultaat. In Engelse teksten wordt deze sectie
% ``Discussion'' genoemd. Had je deze uitkomst verwacht? Zijn er zaken die nog
% niet duidelijk zijn?
% Heeft het onderzoek geleid tot nieuwe vragen die uitnodigen tot verder 
%onderzoek?

This research started with the following research question: In which situations would the usage of headless Drupal be advantageous? From the proof of concept, it is pretty clear that going headless to display something as simple as a list of content requires quite a lot of work, knowledge, and time. As displaying data is the core function of most web applications, it seems that going headless for simple projects like this would be quite a lot of work. This is why it is more interesting to use the front-end functionalities that are included with the CMS, as they do these things perfectly, and only take a few minutes to set up. 


Another difficulty headless brings with it, is the knowledge that is required. Next to the knowledge of the CMS that will be used, there also has to be knowledge of other front-end web frameworks or native mobile development. This means that a lot of different people would be needed for big headless projects, as all of that knowledge is hard to find in one person.

A final drawback to consider is that, when going headless, you lose one of the most important features that make a CMS what it is, namely the ease of changing layouts, styling, and other front-end functionalities. This means that clients will not have access to these things and any changes in layout or design will have to be made by developers. The only way to solve this is by making a separate way of doing these things on the front-end, but that also takes time and resources.

One situation where going headless could prove to be an advantageous approach is when more than one channel will be used to distribute the content in the CMS. When only using the CMS, you have the downside of only being able to create one website, which is tightly coupled to the back-end. Going headless allows you to, for example, create native applications for IOS or Android devices, and also allows you to create multiple websites or web applications that are using the same back-end.

This research has determined in which cases headless might be an interesting approach. For any future research, it would be interesting to poll for the popularity of traditional CMSs in a group of computer science students, to see how well students know them and how much experience they have with them. This way, a strategy could be set up by companies who still mainly use a traditional CMS to attract students by offering headless projects.
