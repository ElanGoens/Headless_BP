%%=============================================================================
%% Inleiding
%%=============================================================================

\chapter{\IfLanguageName{dutch}{Inleiding}{Introduction}}
\label{ch:inleiding}

\section{Context}
A \gls{CMS}, or Content Management System, is, simply put, a piece of sofware that can be used create applications and manage the content used in these applications. There are two big use cases for a Content Management System: Enterprise Content Management (ECM), where the content managed is used in Enterprise applications, and Web Content Management (WCM), where the focus is on Web Applications and websites. The latter is the most common type. This type will also be the focus of this paper.

\gls{Drupal} is a free, open-source CMS written in \gls{PHP}. It was created in 2001 by Dries Buytaert, an open-source software developer. It is often seen as one of the most important players in the open-source web development community, and is widely used in websites and web applications around the world.

Many other CMSs exist, notable examples being \gls{Wordpress} and \gls{Joomla}. The choice to focus on Drupal stems from the fact that, in recent years, Drupal has evolved to be much more than a traditional CMS and has focused a lot more on communication with clients requesting data ~\autocite{So2018}. This is an important quality to have, considering how most users interact with content throught multiple different channels. For example, Facebook users can access the same content throught the web application, as throught the mobile app.

\gls{Headless} (or decouped) is a concept that has been around for quite a while in the Drupal community, but it has started gaining some real traction the last couple of years. The headless usage of a CMS means that the back-end, the actual core of the CMS, is seperated from the integrated front-end it comes with. 

\section{\IfLanguageName{dutch}{Probleemstelling}{Problem Statement}}
\label{sec:probleemstelling}

Research into headless is getting more and more popular. Especially in the Drupal community, developers are realising the importance of modern front-end frameworks and tools and want to combine these with the CMS they know and love. 

Drupal, and other CMSs in general, are also going out of style at colleges and university. There is a severe lack of courses still teaching these subjects. Instead, there are more and more courses teaching modern javascript and mobile frameworks.


This is why it is so important to do research on headless. Businesses that are completely focussed on Drupal are already having a hard time recruiting young talent, and this will only get more difficult in the coming years. This paper can therefore be of vital importance to Dropsolid, to learn more about headless and how to effectively use it in new projects.

\section{\IfLanguageName{dutch}{Onderzoeksvraag}{Research question}}
\label{sec:onderzoeksvraag}

As already mentioned in section 1.1, this research will focus mainly on the usage and possibilities of headless Drupal. The main research question is:
\begin{itemize}
	\item In which situations would the usage of headless Drupal be advantageous?
\end{itemize}

Next to that there are a few supporting research questions:
\begin{itemize}
	\item What is needed to set up headless Drupal?
	\item Is headless a must-have, or a hype?
\end{itemize}

\section{\IfLanguageName{dutch}{Onderzoeksdoelstelling}{Research objective}}
\label{sec:onderzoeksdoelstelling}

This research will be a proof-of-concept to show what possibilities exist in the world of headless. The goal of this research is to give more insight into headless and how to use is, and to give some perspective to companies as to when to use this technology, and when not to.

\section{\IfLanguageName{dutch}{Opzet van deze bachelorproef}{Structure of this bachelor thesis}}
\label{sec:opzet-bachelorproef}

% Het is gebruikelijk aan het einde van de inleiding een overzicht te
% geven van de opbouw van de rest van de tekst. Deze sectie bevat al een aanzet
% die je kan aanvullen/aanpassen in functie van je eigen tekst.

The rest of this thesis will be constructed like this:

Chapter~\ref{ch:stand-van-zaken} will contain an overview of the basics of Drupal and a study of the possibilities headless Drupal has to offer.

Chapter~\ref{ch:methodologie} will discuss the methodology and will illustrate the ways in which headless Drupal will be tested in order to answer the research questions.

In chapter~\ref{ch:proofofconcept} the possibilities of headless Drupal will be studied closely and will be tested out. This will be done in a few different steps:

\begin{enumerate}
	\item Setting up a new Drupal website on a local environment
	\item Adding content to the Drupal website
	\item Exposing the content available on the Drupal website
	\item Consuming the exposed content with the front-end framework Angular
\end{enumerate}

% TODO: Vul hier aan voor je eigen hoofstukken, één of twee zinnen per hoofdstuk

Lastly, in chapter~\ref{ch:conclusie}, a conclusion will be formulated based on what was discovered in chapter~\ref{ch:proofofconcept}. The research questions will also be answered, and this will be the starting point of any future research that has to be done on using a headless CMS.