%%=============================================================================
%% Inleiding
%%=============================================================================

\chapter{\IfLanguageName{dutch}{Inleiding}{Introduction}}
\label{ch:inleiding}

\section{Context}
A CMS, or Content Management System, is, simply put, a piece of sofware that can be used create applications and manage the content used in these applications. There are two big use cases for a Content Management System: Enterprise Content Management (ECM), where the content managed is used in Enterprise applications, and Web Content Management (WCM), where the focus is on Web Applications and websites. The latter is the most common type. This type will also be the focus of this paper.

Drupal is a free, open-source CMS. It was created in 2001 by Dries Buytaert, an open-source software developer. It is often seen as one of the most important players in the open-source web development community, and is widely used in websites and web applications around the world.

Many other CMSs exist, notable examples being Wordpress and Joomla. The choice to focus on Drupal stems from the fact that, in recent years, Drupal has evolved to be much more than a traditional CMS and has focused a lot more on communication with clients requesting data ~\autocite{So2018}. This is an important quality to have, considering how most users interact with content throught multiple different channels. For example, Facebook users can access the same content throught the web application, as throught the mobile app.

Headless (or decoupled) is a concept that has been around for quite a while in the Drupal community, but it has started gaining some real traction the last couple of years. The headless usage of a CMS means that the back-end, the actual core of the CMS, is seperated from the integrated front-end it comes with. 

\section{\IfLanguageName{dutch}{Probleemstelling}{Problem Statement}}
\label{sec:probleemstelling}

Research into headless is getting more and more popular. Especially in the Drupal community, developers are realising the importance of modern front-end frameworks and tools and want to combine these with the CMS they know and love. 

Drupal, and other CMSs in general, are also going out of style at colleges and university. There is a severe lack of courses still teaching these subjects. Instead, there are more and more courses teaching modern javascript and mobile frameworks.

This is why it is so important to do research on headless. Businesses that are completely focussed on Drupal are already having a hard time recruiting young talent, and this will only get more difficult in the coming years. This paper can therefore be of vital importance to Dropsolid, to learn more about headless and how to effectively use it in new projects.

\section{\IfLanguageName{dutch}{Onderzoeksvraag}{Research question}}
\label{sec:onderzoeksvraag}

As already mentioned in section 1.1, this research will focus mainly on the usage and possibilities of headless Drupal. The main research question is:
\begin{itemize}
	\item How can headless Drupal be used in a profitable way? 
\end{itemize}

Next to that there are a few supporting research questions:
\begin{itemize}
	\item How can headless Drupal help attract young developers?
	\item What is needed to set up headless Drupal?
	\item In which situations would the usage of headless Drupal be advantageous?
	\item How far along is Drupal in headless development compared to other traditional CMSs?
\end{itemize}

\section{\IfLanguageName{dutch}{Onderzoeksdoelstelling}{Research objective}}
\label{sec:onderzoeksdoelstelling}

This research will be a proof-of-concept to show what possibilities exist in the world of headless. The goal of this research is to give more insight into headless and how to use is, and to give some perspective to companies as to when to use this technology, and when not to.

\section{\IfLanguageName{dutch}{Opzet van deze bachelorproef}{Structure of this bachelor thesis}}
\label{sec:opzet-bachelorproef}

% Het is gebruikelijk aan het einde van de inleiding een overzicht te
% geven van de opbouw van de rest van de tekst. Deze sectie bevat al een aanzet
% die je kan aanvullen/aanpassen in functie van je eigen tekst.

De rest van deze bachelorproef is als volgt opgebouwd:

In Hoofdstuk~\ref{ch:stand-van-zaken} wordt een overzicht gegeven van de stand van zaken binnen het onderzoeksdomein, op basis van een literatuurstudie.

In Hoofdstuk~\ref{ch:methodologie} wordt de methodologie toegelicht en worden de gebruikte onderzoekstechnieken besproken om een antwoord te kunnen formuleren op de onderzoeksvragen.

% TODO: Vul hier aan voor je eigen hoofstukken, één of twee zinnen per hoofdstuk

In Hoofdstuk~\ref{ch:conclusie}, tenslotte, wordt de conclusie gegeven en een antwoord geformuleerd op de onderzoeksvragen. Daarbij wordt ook een aanzet gegeven voor toekomstig onderzoek binnen dit domein.