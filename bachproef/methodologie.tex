%%=============================================================================
%% Methodologie
%%=============================================================================

\chapter{\IfLanguageName{dutch}{Methodologie}{Methodology}}
\label{ch:methodologie}

%% TODO: Hoe ben je te werk gegaan? Verdeel je onderzoek in grote fasen, en
%% licht in elke fase toe welke stappen je gevolgd hebt. Verantwoord waarom je
%% op deze manier te werk gegaan bent. Je moet kunnen aantonen dat je de best
%% mogelijke manier toegepast hebt om een antwoord te vinden op de
%% onderzoeksvraag.


This proof of concept explains in more detail how a headless Drupal site can be set up, and how it can be used. It has been constructed in three distinct phases: 
\begin{enumerate}
	\item  Phase one: Setting up a standard Drupal site
	\item  Phase two: Exposing the data available on the Drupal site
	\item  Phase three: Consuming the data exposed on the Drupal site
\end{enumerate}

\section{Phase one: Setting up a new Drupal site}

This phase contains a basic explanation of how to install a new Drupal 9 site. It also dives a little deeper into the creation of content types, and gives some background on how to configure your site.

\section{Phase two: Exposing the data available on the Drupal site}

Phase two contains an in depth explanation on how to use the two ways for exposing data in Drupal 9 mentioned in the previous chapter: JSON:API and REST Web Services. Both of these methods are tested out, and the benefits and pitfalls of both are analyzed.

\section{Phase three: Consuming the data exposed on the Drupal site}

Finally, the last phase gives some more insight on how a JavaScript front-end framework can be used to consume the data from the Drupal site.