%%=============================================================================
%% Samenvatting
%%=============================================================================

% TODO: De "abstract" of samenvatting is een kernachtige (~ 1 blz. voor een
% thesis) synthese van het document.
%
% Deze aspecten moeten zeker aan bod komen:
% - Context: waarom is dit werk belangrijk?
% - Nood: waarom moest dit onderzocht worden?
% - Taak: wat heb je precies gedaan?
% - Object: wat staat in dit document geschreven?
% - Resultaat: wat was het resultaat?
% - Conclusie: wat is/zijn de belangrijkste conclusie(s)?
% - Perspectief: blijven er nog vragen open die in de toekomst nog kunnen
%    onderzocht worden? Wat is een mogelijk vervolg voor jouw onderzoek?
%
% LET OP! Een samenvatting is GEEN voorwoord!

%%---------- Nederlandse samenvatting -----------------------------------------
%
% TODO: Als je je bachelorproef in het Engels schrijft, moet je eerst een
% Nederlandse samenvatting invoegen. Haal daarvoor onderstaande code uit
% commentaar.
% Wie zijn bachelorproef in het Nederlands schrijft, kan dit negeren, de inhoud
% wordt niet in het document ingevoegd.

\IfLanguageName{english}{%
\selectlanguage{dutch}
\chapter*{Samenvatting}
Dit onderzoek heeft als doel een inzicht te bieden in het headless gebruik van een traditioneel CMS. De reden waarom dit onderzoek zo belangrijk is spreekt voor zichzelf: de populariteit van traditionele content management systems zoals Drupal neemt de laatste jaren erg af, ten voordele van nieuwere technologieën zoals javascript en mobile frameworks. Daarom is het belangrijk voor bedrijven die voornamelijk met een CMS bezig zijn om headless meer en meer aandacht te schenken, anders kan deze evolutie wel eens het einde van de traditionele CMS betekenen.

In dit onderzoek wordt gekeken naar een van de meest populaire content management systems, namelijk Drupal, en de innovaties die er gedaan zijn om headless zo toegankelijk mogelijk te maken. Er wordt onderzocht in welke situaties headless een voordeel kan geven, maar ook wanneer het overbodig is. Daarnaast wordt er een proof of concept opgezet om aan te tonen wat headless precies inhoud.
\selectlanguage{english}
}{}

%%---------- Samenvatting -----------------------------------------------------
% De samenvatting in de hoofdtaal van het document

\chapter*{\IfLanguageName{dutch}{Samenvatting}{Abstract}}
This research has the goal to give more insight into the headless use of a traditional CMS. The reason why this research is so important speaks for itself: the popularity of traditional content management systems like Drupal has been fading the last couple of years, in favor of new technologies like javascript and mobile frameworks. This is the reason why it is important for companies who primarily use a CMS, to give more attention to headless, otherwise this evolution could mean the end of the traditional CMS.

This research will take a closer look at one of the most popular contentmanagement systems, Drupal, and which innovations it has made to make headless as accesible as possible. It will take a look at different situations and determine when headless might be favourable, but also when it is unnecesary. Next to that, a proof of concept will be made to show what headless actually is, and what it could be.
